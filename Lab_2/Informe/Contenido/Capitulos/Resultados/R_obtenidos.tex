Los resultados de la implementación y sus requisitos especifíficos fueron resumidos 
en el cuadro \ref{tab:Resultados} junto con cuales se completaron, el puntaje de 
funcionamiento según la pauta dada de autoevaluación (entre 0 y 1); las pruebas 
realizadas fueron consultas a la base de conocimientos en el archivo de pruebas que se subirá 
a la plataforma github y uvirtual.\\

Con respecto a la posible razón de los fallos en el requisito 18 (Descomprimir una imagen), 
se debe a que, si bien se recuperan los pixeles con el color más usados en la posición (x,y) respectiva,
se da por hecho que la profundidad de estos es 0 por lo que funcionaría bien en imagenes bidimensionales 
pero no en las que se usaron; una idea para solucionar esto sería que, al comprimir la imagen,
se almacene también una lista con las profundidades en orden, pero, la razón por la que no se quiso implementar 
esto, es que, si tenemos que almacenar una lista completa para comprimir una imagen, pierde un poco el sentido de hacerlo, 
ya que, aún se estaría ocupando un gran espacio.