Para concluir, cabe hablar acerca de como fue abordar el proyecto haciendo uso de este paradigma; si bien no se tuvieron las facilidades que dan las variables mutables 
 del paradigma imperativo procedural, ni las funciones del paradigma funcional visto en el laboratorio pasado, los alcances de este paradigma se ven también como ilimitados, 
 ya que, al no tener que pensar tanto en los pasos a seguir para resolver el problema, sino más bien pensar en el problema y sus caracteristicas en sí; dejar que el interprete lo resuelva 
 haciendo uso de sus herramientas ,que no necesitamos conocer, simplifica mucho la resolución del problema; es más, bastó con tener un buen entendimiento del mismo.\\

 Finalmente a modo de opinión personal, el paradigma lógico se me hizo muy interesante, es más, fue entretenida la parte de programar durante el laboratorio haciendo uso de este 
 paradigma, incluso más que en el paradigma funcional, y espero poder seguir aprendiendo del mismo por mi cuenta.