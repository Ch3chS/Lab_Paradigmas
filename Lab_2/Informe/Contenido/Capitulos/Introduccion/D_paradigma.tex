Para la resolución del problema planteado anteriormente, se hará uso de un lenguaje
 de programación llamado Prolog el cual se encuentra enfocado en el paradigma de programación lógica, ya teniendo en cuenta
   esto podría surgirnos la pregunta "¿Que es el paradigma lógico?" 

   \subsubsection{Paradigma de programación lógica}
    El paradigma lógico es aquel donde se hace uso de la lógica que subyace a un problema para su posterior resolución, 
    esto a través de un conjunto de fórmulas lógicas que suelen verse, al llevar esto a la programación, como predicados o cláusulas de Horn\cite{logica}. 
    Lo anterior convierte la resolución de un problema en la búsqueda de la lógica detrás del mismo, y no en una lista de pasos a seguir como 
    acostumbramos en otros paradigmas; esto a través de una base de conocimientos con hechos y reglas, haciendo uso también de la unificación.

    