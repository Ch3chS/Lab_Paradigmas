Comenzando ahora con el diseño de la solución para la problemática de este semestre, se realizará la definición y especificación
de los distintos TDAs para luego ver como puede ser la posible implementación de los mismos y lo que se realizará para los problemas
específicos.

\subsection{TDAs}
 Empezando con los TDAs el enfoque será que una imagen esta compuesta de pixeles los cuales
  son de alguno de los siguientes 3 tipos, pixbit-d, pixrgb-d o pixhex-d; haciendo los TDAs de estos 3 tipos
  de pixel junto con el TDA imagen podríamos lograr la mayoría de los requisitos específicos sin problema, sin 
  embargo para un manejo de ciertos requisitos específicos como el de que se pase una imagen de color RGB a hexadecimal
  que no entraría en ninguno de los TDAs de pixeles específicamente, podríamos definir un quinto TDA "pixeles" donde hayan
  funciones de 2 o más tipo de pixeles, o, en otro caso, hagan uso de un conjunto de pixeles del mismo tipo. Finalmente se definiría
  un último TDA histogram para el requisito con el mismo nombre.\\

  \subsubsection{Definición}
    Ya sabiendo cuales son los TDA a implementar se procederá con su definición:
    \begin{itemize}
      \item TDA image: Una imagen será aquel conjunto de pixeles con un ancho y un alto específicos
        que puede estar o no comprimida, por lo cual, se debe poder saber cual es el color más usado en la misma.
      \item TDA pixeles: Será aquel conjunto con bajo grado de especificación donde se podrán realizar operaciones con 2 o más tipos de pixeles, ó,
        con 2 o más pixeles del mismo tipo.
      \item TDA pixbit-d: Será aquel pixel con una posición específica en una imagén de 3 dimensiones con 2 posibles colores, blanco o negro.
      \item TDA pixrgb-d: Será aquel pixel con una posición específica en una imagén de 3 dimensiones con color de tipo rgb.
      \item TDA pixhex-d: Será aquel pixel con una posición específica en una imagén de 3 dimensiones con color de tipo hexadecimal.
      \item TDA histogram: Será el gráfico de cuantos pixeles de cada color hay contenidos en una imagén
    \end{itemize}

  \subsubsection{Especificación}
    La especificación general se puede encontrar en el cuadro \ref{tab:TDAs} de los anexos;
    si se desean tener más detalles se recomienda ver el código de cada TDA donde se específica
    de manera más detallada debido al poco espacio de este medio.

  \subsubsection{Implementación}
    La implementación de una imagen será con una lista que contendrá 2 enteros positivos que son el ancho y el alto de la misma, junto
    con una lista con los diferentes pixeles que tiene; por último tendrá un elemento con el color más
    usado en la imagen, cuyo tipo dependerá del tipo de imagen que sea (bitmap, pixmap o hexmap).\\

    Los pixeles en sí estarán un poco más adelante, pero algo que los 3 tipos de estos tendrán en común, es que serán implementados con listas 
    donde el primer y segundo elemento corresponden a la posición (x,y) del pixel y el último elemento de la lista
    corresponderá a la profundidad del pixel, donde, estos 3 valores mencionados, son enteros positivos.\\

    La gran diferencia se encontrará en los datos que hay entre medio de los valores (x,y) y depth (profundidad)
    los cuales corresponden al color del pixel; los TDA de pixeles quedarían de la siguiente forma:
    \begin{itemize}
        \item Pixbit-d: Este tipo de pixel tendrá un bit entre medio de tipo entero, el cual solo podrá 
        tomar los valores 0 y 1. De esta forma un pixbit-d será una lista de la forma:
        \begin{equation*}
            [int, int, int, int] = [x \geq 0, y \geq 0, bit(0|1), depth \geq 0]
        \end{equation*}
        
        \item Pixrgb-d: Este tipo de pixel tendrá además una lista de 3 enteros entre 0 y 255 que representarán la 
        cantidad de cada color que tiene el pixel siendo r = rojo (red), g = verde (green) y b = azul (blue)
        quedandonos una lista de la forma:
        \begin{equation*}
            [int, int, [int, int, int], int] = [x \geq 0, y \geq 0, [0 \leq r \leq 255, 0 \leq g \leq 255, 0 \leq b \leq 255], depth \geq 0]
        \end{equation*}
        
        \item Pixhex-d: Este tipo de pixel tendrá un string con formato \#XXXXXX donde las X pueden tomar valores de un dígito hexadecimal (entre 0 y F(15))
        donde las 2 primeras X es la cantidad de rojo, las 2 de al medio son de verde y las 2 últimas son de azul; la representación quedaría de la forma:
        \begin{equation*}
            [int, int, string, int] = [x \geq 0, y \geq 0, hex, depth \geq 0]
        \end{equation*}
    \end{itemize}

    Por otro lado el histograma será una lista de listas de 2 elementos, en estas listas de 2 la cabeza será un color
    y el segundo elemento la cantidad de veces que ese color se repite.
