Ya entendiendo  por encima el problema y el paradigma a utilizar, para solucionarlo profundizaremos
 en el mismo, ya que, conociendo bien el problema y sus requisitos específicos llegaremos
 a una mejor solución.

\begin{enumerate}
    \item TDAs: El primer desafio será diseñar los tipos de datos abstractos que nos permitan alcanzar la solución de la manera más
    eficiente posible, por lo que este primer requisito es más que nada de diseño de la solución que estará en la sección \ref{sec:diseño}.
    \item Contructor de imágenes: Ya habiendo mencionado los TDAs el que principalmente se llevará nuestra atención será aquel que 
    represente una imagen como tal, ya que, al implementarlo, debemos tener diferentes consideraciones como:
    \begin{itemize}
        \item Tipo de imagen: En un principio se deberían tener 3 tipos de imagen correspondientes a bitmap, pixmap y hexmax y que
        el programa pueda reconocer de que tipo se trata; la implementación de lo antes mencionado se verá en la siguiente sección.
        \item Debe ser posible también poder cambiar entre imágenes tipo pixmap y hexmap (RGB y hexadecimal).
        \item ¿Está comprimida?: Es necesario saber si una imagen está o no comprimida
        \item Comprimir y descomprimir: Como ya se dijo que una imagen puede estar o no comprimida entonces deben haber funciones que compriman
     y descompriman la imagen como tal, dependiendo de la implementación podría verse también como requisito una función que nos diga 
     cuál es el color más usado en la imagen (Histograma).
    \end{itemize}
    
    \item Cambiar dirección o rotar: Otro requisito especifífico a abordar, es dar la posibilidad de invetir la imagen ya sea en el eje X o Y, 
    además de poder rotar esta en 90° hacia la derecha.
    \item Recortar: También se debe dar la posibilidad de recortar la imagen para obtener una nueva imagen de un subsector de la original.
    \item Invertir colores: Como el nombre lo dice, se debe ser capaz de invertir los colores de imágenes RGB.
    \item Imagen a string: Debe también cubrirse la posibilidad de que la imagen pueda ser representada como un conjunto de caracteres.
    \item Separar en capas: La imagen debe tener profundidad, y para esto, se deben tener varias capas entre las que le es posible separarse.
\end{enumerate}
