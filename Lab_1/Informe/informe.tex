\documentclass[10pt,letterpaper,openany]{article}
\usepackage[utf8]{inputenc}
\usepackage{csquotes}
\usepackage[left=2.54cm, right=2.54cm, top=2.54cm, bottom=2.54cm]{geometry}
\usepackage[spanish]{babel}
\usepackage{amsmath}
\usepackage{amsfonts}
\usepackage{amssymb}
\usepackage{graphicx}
\usepackage{lipsum}
\usepackage{apacite}

\title{Informe}
\author{Sergio Espinoza}

\begin{document}
    \begin{titlepage}
    \begin{center}
        {\LARGE \textbf{Universidad de santiago de Chile}}\\
        \vspace{0.25cm}
        {\large \textbf{Labor L\ae titia nostra}}
        \vspace{1cm}
        \begin{figure}[h]
            \centering
            \includegraphics[scale=1.5]{Contenido/Imagenes/Logo color.png}
        \end{figure}
        
        {\Large  Informe para el ramo Paradigmas de programación}
        \vspace{0.3cm}
        \rule{15cm}{0.5mm}
        {\Large Laboratorio 1: Paradigma Funcional\\ Editor de imágenes usando Scheme, DrRacket}
        \rule{15cm}{0.5mm}
        \vspace{2cm}
        \\
        {\Large \textbf{Sergio Osvaldo Andres Espinoza Gonzalez}}\\
        \vspace{0.5cm}
        {\Large \textbf{Facultad de ingeniería}}
        \vfill
        {\Huge \textbf{Agosto de 2022}}
    \end{center}
\end{titlepage}
    
    
    \tableofcontents
    \newpage

    %------------------------------------------------------------------------

    
    \section{Introducción}
    En este informe se abordará el laboratorio número 1 del ramo paradigmas de programación
    con el formato expuesto en el indice del mismo. A continuación se darán breves descripciones
    de los dos grandes pilares a ver para encontrar la posterior resolución.  
        \subsection{Descripción del problema}
            Para empezar debemos conocer el problema a resolver, el cual es la creación de un
 software que permita la edición de imágenes de manera similar al software de código
  abierto GIMP o el de pago conocido como Photoshop; la finalidad del proyecto, es 
  poder realizar las funcionalidades de un editor como los antes mencionados sin la necesidad 
  de una interfaz gráfica.
        \subsection{Descripción del paradigma}
            Para la resolución del problema planteado anteriormente, se hará uso de un lenguaje
 de programación llamado Racket el cual es un derivado de Scheme que se encuentra
  enfocado en el paradigma de programación funcional, ya teniendo en cuenta
   esto podría surgirnos la pregunta "¿Que es el paradigma funcional?" 

   \subsubsection{Paradigma funcional}
   El paradigma funcional es aquel que, como dice su nombre, hace uso de funciones y
   todo lo relacionado con las mismas, es decir, funciones de orden superior, evaluación
    de funciones en tiempo real, currificación y más.\cite{funcional}

    %Mejorar este parrafo

    %------------------------------------------------------------------------

    \section{Análisis del problema}
        Ya entendiendo  por encima el problema y el paradigma a utilizar para solucionarlo profundizaremos
 en el mismo, ya que, conociendo bien el problema y sus requisitos especifíficos llegaremos
 a una mejor solución.

\begin{enumerate}
    \item TDAs: El primer desafio será diseñar los tipos de datos abstractos que nos permitan alcanzar la solución de la manera más
    eficiente posible, por lo que este primer requisito es más que nada de diseño de la solución.
    \item Contructor de imágenes: Ya habiendo mencionado los TDAs el que principalmente se llevará nuestra atención será aquel que 
    represente una imagen como tal, ya que, al implementarlo debemos tener diferentes consideraciones como:
    \begin{itemize}
        \item Tipo de imagen: En un principio se deberían tener 3 tipos de imagen correspondientes a bitmap, pixmap y hexmax y que
        el programa pueda reconocer de que tipo se trata; la implementación de lo antes mencionado se verá en la siguiente sección.
        \item Debe ser posible también poder cambiar entre imágenes tipo pixmap y hexmap (RGB y hexadecimal).
        \item ¿Está comprimida?: Es necesario si una imagen está o no comprimida
        \item Comprimir y descomprimir: Como ya se dijo que una imagen puede estar o no comprimida entonces deben haber funciones que compriman
     y descompriman la imagen como tal, dependiendo de la implementación podría verse también como requisito una función que nos diga 
     cuál es el color más usado en la imagen (Histograma).
    \end{itemize}
    
    \item Cambiar dirección o rotar: Otro requisito especifífico a abordar es dar la posibilidad de invetir la imagen ya sea en el eje X o Y, 
    además de poder rotar esta en 90° hacia la derecha o izquierda.
    \item Recortar: También se debe dar la posibilidad de recortar la imagen para obtener una nueva imagen de un subsector de la original.
    \item Invertir colores: Como el nombre lo dice, se debe ser capaz de inverit los colores independiente del tipo de imagen.
    \item Ajustar canal: Debe también darse la posibilidad de ajustar el canal de una imagen con pixeles RGB-D (con profundidad).
    \item Imagen a string: Debe también cubrirse la posibilidad de que la imagen pueda ser representada como un conjunto de caracteres.
    \item Separar en capas: La imagen debe tener profundidad, y para esto, se deben tener varias capas entre las que le es posible separarse.
\end{enumerate}


    \section{Diseño de la solución}
        Comenzando ahora con el diseño de la solución para la problemática de este semestre
 deberiamos partir definiendo que tipos de dato abstracto o TDAs, los cuales serán los que 
 nos permitirán una mejor representación para este editor de imagenes.

 \subsection{TDAs e implementación}
 El objeto central con el que trabajaremos en un editor de imagenes es bastante predecible, 
 las imagenes, pero a su vez bien es sabido que las imagenes están subdivididas en 
  pixeles con los que trabajar si se quieren realizar
 cambios en esta; así que, para este trabajo definiremos 3 tipos de pixeles y su implementación para que 
 podamos trabajar correctamente las imagenes en un archivo principal.\\

Los pixeles en sí estarán un poco más adelante, pero algo que los 3 tipos de estos tendrán en común, es que serán implementados con listas 
donde los 2 primeros elementos corresponden a la posición (x,y) del pixel y el último elemento de la lista
 corresponderá a la profundidad del pixel, donde, estos 3 valores mencionados son enteros positivos.\\

 La gran diferencia se encontrará en los datos que hay entre medio de los valores (x,y) y depth (profundidad) que
 serán de la siguiente forma:
 \begin{itemize}
    \item Pixbit-d: Este tipo de pixel tendrá un bit entre medio de tipo entero, el cual solo podrá 
    tomar los valores 0 y 1. De esta forma un pixbit-d será una lista de la forma
    \begin{equation*}
        '(int, int, int, int) = (x \geq 0, y \geq 0, bit(0|1), depth \geq 0) 
    \end{equation*}
     
    \item Pixrgb-d: Este tipo de pixel tendrá además tendrá 3 enteros entre 0 y 255 que representarán la 
    cantidad de cada color que tiene el pixel siendo r = rojo (red), g = verde (green) y b = azul (green)
    quedandonos una lista de la forma
    \begin{equation*}
        '(int, int, int, int, int, int) = (x \geq 0, y \geq 0, 0 \leq r \leq 255, 0 \leq g \leq 255, 0 \leq b \leq 255, depth \geq 0) 
    \end{equation*}
    
    \item Pixhex-d: Este tipo de pixel tendrá un string con formato \#XXXXXX donde las X pueden tomar valores de un dígito hexadecimal (entre 0 y F(15))
    donde las 2 primeras X es la cantidad de rojo, las 2 de al medio son de verde y las 2 últimas son de azul; la representación quedaría de la forma
    \begin{equation*}
        '(int, int, string, int) = (x \geq 0, y \geq 0, hex, depth \geq 0) 
    \end{equation*}
 \end{itemize}

 presentar su enfoque de solución, describir, diagramar, 
 descomposición de problemas, algoritmos o técnicas empleados para
  problemas particulares, recursos empleados
    (presentar su enfoque de solución, describir, diagramar, descomposición de problemas, algoritmos o técnicas empleados para problemas particulares, recursos empleados) (max. 2.5 páginas. Puede incluir diagramas como anexos fuera del límite de las 10 páginas) (30\%)

    \section{Consideraciones de implementación}
    (estructura del proyecto, bibliotecas empleadas, razones de su elección, compilador o interprete usado, etc.) (max 1 página) (10\%)

    \section{Instrucciones de uso}
    (ejemplos, resultados esperados, posibles errores) (max 1 página. Complementar con ejemplos detallados en anexos que están fuera del límite de 10 páginas) (15\%)

    \section{Resultados obtenidos}
    Esto puede abordarse listando todos los requerimientos del proyecto en una tabla indicando el grado de alcance. Indicar que tipos de pruebas se hicieron. Cuantas de las pruebas fueron exitosas, cuantas fracasaron, razones de fallos. Especificar que funciones no se completaron y el porque no se completaron. (máximo 1 página) (10\%)

    \section{Evaluación completa}
    \cite{einstein}

    \section{Conclusión}
    (respecto de los alcances, limitaciones, dificultades de usar el paradigma para abordar el problema. Para los informes 2 en adelante, contrastar resultados alcanzados con el paradigma de turno con los anteriores) (máximo 1 página) (10\%)

    \newpage
    \bibliographystyle{apacite}
    \bibliography{Bibliografia.bib}
    
    \section*{Anexos}
        \input{Anexos.tex}

\end{document}