En un principio las funciones se harán pensando en que no es necesario que se ingresen todos 
los pixeles a usar dentro de una imagen, ya que, se puede rellenar de ser necesario; ni que 
tengan un orden en concreto al ingresarse siempre y cuando podamos saber su posición con sus coordenadas 
(x,y); debido a esto no hay muchas instrucciones de uso como tal, más allá de que se cumpla el formato 
de los ejemplos del cuadro \ref{tab:Ejemplos}, aunque claro, hay muchos más ejemplos en el archivo de pruebas.

Tener en cuenta que:
\begin{itemize}
    \item img1: (image 2 1 (pixrgb-d 0 0 0 140 255 0) (pixbit-d 1 0 140 0 255 0))
    \item img2: (image 1 2 (pixbit-d 0 0 0 0) (pixbit-d 0 1 0 0))
\end{itemize}
Las imagenes de ejemplo son pequeñas para que quepan en la tabla