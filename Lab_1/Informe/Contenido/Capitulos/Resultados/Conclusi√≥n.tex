Partiendo bajo la premisa de que practicamente todo se puede representar con funciones, el alcance 
de este paradigma se vuelve casi ilimitado, pero, como bien es sabido, no tenemos la costumbre de usarlo. 
Debido a lo anterior pudieron haber sido un poco lentos los avances en un principio, esto porque a pesar de poder usar 
funciones en el imperativo procedural al que estamos tan acostumbrados, se suele usar variables mutables en estas, en cambio, el paradigma 
de programación funcional directamente no posee variables como tal, lo que puede ser una importante limitación para 
alguien que viene del imperativo procedural.\\

Finalmente destacaría que, a pesar de no tener la comodidad de usar cosas que siempre se han usado en los ramos anteriores de la carrera, este 
es un paradigma de programación muy interesante y que me divertí bastante practicandolo en DrRacket para la elaboración de este laboratorio y espero 
aprender más de el en algun futuro.